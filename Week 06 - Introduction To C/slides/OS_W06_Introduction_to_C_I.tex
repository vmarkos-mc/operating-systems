% !TeX spellcheck = en_GB
% !TeX TS-program = xelatex
\documentclass[aspectratio=169, 12pt]{beamer}
\usefonttheme{professionalfonts}
\usefonttheme{serif}
\usepackage[T1]{fontenc}
\usepackage{fontspec-xetex}

\usepackage{booktabs}
\usepackage{listings}
\usepackage{subcaption}
\setmainfont{Lato}

% Bibliography
\usepackage[style=authortitle,backend=biber]{biblatex}
\addbibresource{references.bib}

% Local configuration
\renewcommand{\figurename}{}
\DeclareCaptionFormat{custom}
{%
	\tiny #3
}
\captionsetup{format=custom}

% Title stuff
\title{Operating Systems}
\subtitle{A (Soft) Introduction to C, Part I}
\date{Week 06}
\author{Vassilis Markos, Mediterranean College}

\usetheme{streamline}

% Local Commands
\newcommand{\ohref}[1]{\href{#1}{\texttt{#1}}}

% Code listings

\definecolor{codegreen}{rgb}{0,0.6,0}
\definecolor{codegray}{rgb}{0.5,0.5,0.5}
\definecolor{codepurple}{rgb}{0.58,0,0.82}
\definecolor{backcolour}{rgb}{0.95,0.95,0.92}

\lstdefinestyle{mystyle}{
	backgroundcolor=\color{backcolour},   
	commentstyle=\color{codegreen},
	keywordstyle=\color{magenta},
	numberstyle=\tiny\color{codegray},
	stringstyle=\color{codepurple},
	basicstyle=\ttfamily\footnotesize,
	breakatwhitespace=false,         
	breaklines=true,                 
	captionpos=b,                    
	keepspaces=true,                 
	numbers=left,                    
	numbersep=5pt,                  
	showspaces=false,                
	showstringspaces=false,
	showtabs=false,                  
	tabsize=2
}

\lstset{style=mystyle}

\begin{document}

	\begin{frame}
		\titlepage
	\end{frame}

	\begin{frame}{Contents}
		\tableofcontents
	\end{frame}
	
	\section{Introduction to C++: \texttt{helloWorld.c}}
	
	\sectionframe	
	
	\begin{frame}{Some Programming Humour, First\ldots}
		\begin{figure}[!htb]
			\centering
			\includegraphics[height=0.5\textheight]{./assets/pointers.png}
			\caption{If you already understand this, you might be in the wrong room. Source: \href{https://xkcd.com/138/}{\texttt{https://xkcd.com/138/}}.}
		\end{figure}
	\end{frame}

	\begin{frame}[fragile]{\texttt{helloWorld.c}}
		\lstinputlisting[language=C]{../source/helloWorld.c}
		To execute: \texttt{gcc helloWorld.c -o helloWorld \&\& ./helloWorld}
	\end{frame}

	\begin{frame}{Notes on \texttt{helloWorld.c}}
		\begin{itemize}
			\item \texttt{\#include} is a Preprocessor command informing the preprocessor that it should ``copy--paste'' into this file the contents of the \texttt{stdio} library.
			\item Printing and IO, in general, is not built into C, as it is, e.g., in Python. So, we have to import \texttt{stdio.h} for that purpose.
			\item \texttt{return 0;} is a typical spell included in the end of every \texttt{main()} function.
			\begin{itemize}
				\item We shall see in the near future that we can safely forget this in many cases, though.
			\end{itemize}
		\end{itemize}
	\end{frame}

	\begin{frame}{Fancy String Characters in C++}
		\small\centering%
		\begin{tabular}{cl}
			\toprule%
			\textbf{Sequence} & \textbf{Meaning} \\\midrule
			\texttt{\textbackslash a}   & System bell (``beeps'')\\%
			\texttt{\textbackslash b}   & Backspace\\%
			\texttt{\textbackslash f}   & Page break (form feed)\\%
			\texttt{\textbackslash n}   & Line break (newline)\\%
			\texttt{\textbackslash r}   & Carriage return (returns the cursor to start of line)\\%
			\texttt{\textbackslash t}   & Tab\\%
			\texttt{\textbackslash\textbackslash}   & Backslash\\%
			\texttt{\textbackslash'}   & Single quote\\%
			\texttt{\textbackslash"}   & Double quote\\%
			\texttt{\textbackslash c}   & Character represented by integer $c$\\%
			\bottomrule
		\end{tabular}
	\end{frame}

	\begin{frame}{Frequently Used C Data Types}
		\centering
		\begin{tabular}{lp{0.6\textwidth}l}
			\toprule%
			\textbf{Type Name} & \textbf{Description} & \textbf{Size}\\\midrule%
			\texttt{char} & Single text character, indicated with single quotes & 1 byte\\%
			\texttt{int} & Signed or unsigned integer & 4 bytes\\%
			\texttt{bool}\footnotemark & Boolean & 1 byte\\%
			\texttt{float} & Float number, 7 decimal digits accuracy & 4 bytes\\%
			\texttt{double} & Double accuracy float, 15 decimal digits. & 8 bytes\\%
			\bottomrule
		\end{tabular}
		\footnotetext{Booleans are not a built--in feature of C; we need to include \texttt{stdbool.h}. Or just use \texttt{0} for \texttt{false} and \texttt{1} for \texttt{true}, alongside bitwise operators --- your choice.\vspace{1em}}
	\end{frame}

	\begin{frame}{Frequently Used C++ Operators}
		\centering
		\begin{tabular}{lp{0.7\textwidth}}
			\toprule%
			\textbf{Operator(s)} & \textbf{Description}\\\midrule%
			\texttt{+}, \texttt{-}, \texttt{*}, \texttt{/} & Addition, subtraction, multiplication, division, priority determined as in maths (plus parentheses).\\%
			\texttt{\%} & Modulus operator: Remainder of integer division, e.g., \texttt{13 \% 4} evaluates to \texttt{1}.\\%
			\texttt{\&\&}, \texttt{||}, \texttt{!} & Logical \texttt{AND}, \texttt{OR}, \texttt{NOT}.\\%
			\texttt{==}, \texttt{!=} & \texttt{equals} and \texttt{not equals}.\\%
			\texttt{<}, \texttt{>} & \texttt{larger than} and \texttt{less than}.\\%
			\bottomrule
		\end{tabular}
	\end{frame}

	\section{Flow Control}
	
	\sectionframe	

	\begin{frame}{Flow Control in C++}
		\centering
		\scalebox{0.75}{%
		\lstinputlisting[language=C]{../source/flowControl.c}}
	\end{frame}

	\begin{headsup}{User Input In C}
		\begin{itemize}
			\item To read user input, we used the following line of C code:
			\lstinputlisting[language=C, linerange={7-7}]{../source/flowControl.c}
			\item Breaking it down a bit:
			\begin{itemize}
				\item \texttt{scanf()} is the \texttt{stdio.h} function that lets us read from the input stream.
				\item The first argument is a \emph{format specifier}, which explains to the compiler how to handle the provided string (array of characters).
				\item The second argument is a pointer to the variable we want this piece of information to be stored at.
				\item For the time being, you can think of the second argument as the target variable, prefixed by an ampersand (\texttt{\&}).
			\end{itemize}
		\end{itemize}
	\end{headsup}

	\begin{headsup}{Format Specifiers}
		\begin{itemize}
			\item The general shape of a format specifier is: \texttt{\%[*][width][length]specifier}.
			\item Some common format specifiers are:
			\begin{itemize}
				\item \texttt{\%d}: Signed integers, with any number of digits.
				\item \texttt{\%u}: Unsigned integers, with any number of digits.
				\item \texttt{\%f}: Floating point numbers, with any number of digits.
				\item \texttt{\%c}: Characters (no null termination here, see below).
				\item \texttt{\%s}: Null--terminated strings, including any character besides whitespace --- so, the scanner stops at the first whitespace, introducing a \texttt{null} character there.
			\end{itemize}
		\end{itemize}
	\end{headsup}

	\begin{headsup}{Format Specifiers}
		\begin{itemize}
			\item In our case, we used a somewhat complex format specifier: \texttt{" \%c"}.
			\item The \texttt{\%c} part in the above is to read the next character entered by the user, as discussed above.
			\item Trying to remove the single blank space on the left, recompiling and executing the executable will result to actually prohibiting you from providing any input.
			\item That is because we use a \emph{newline} character at the end of \texttt{printf()}, which is directly consumed by \texttt{scanf()} next.
			\item To prohibit that from happening, we add a blank space, indicating that \texttt{scanf()} should ignore any trailing whitespace.
		\end{itemize}
	\end{headsup}

	\begin{irrelevant}{Format Specifier Fun}
		\begin{itemize}
			\item A nice resource on format specifiers in C / C++ can be found here: \href{https://cplusplus.com/reference/cstdio/scanf/}{https://cplusplus.com/reference/cstdio/scanf/}.
			\item The format specification language is rich enough to be Turing complete\footfullcite{printfTuring}, i.e., to allow for anything programmable to be programmed on it!
			\begin{itemize}
				\item Available online \href{https://www.usenix.org/system/files/conference/usenixsecurity15/sec15-paper-carlini.pdf}{here}.
			\end{itemize}
			\item An example of implementing Tic--Tac--Toe using just \texttt{printf()}: \href{https://www.ioccc.org/2020/carlini/index.html}{https://www.ioccc.org/2020/carlini/index.html}
		\end{itemize}
	\end{irrelevant}

	\begin{frame}{\texttt{while} Loops in C}
		\centering
		\scalebox{0.75}{%
			\lstinputlisting[language=C]{../source/whileLoop.c}}
	\end{frame}

	\begin{frame}{\texttt{for} Loops in C++}
		\centering
		\scalebox{0.8}{%
			\lstinputlisting[language=C]{../source/forLoop.c}}
	\end{frame}
	
	\begin{frame}{Useful Resources}
		Some resources you might find useful in your C++ journey:
		\begin{itemize}
			\item C Video Tutorial: \href{https://www.youtube.com/watch?v=xND0t1pr3KY}{https://www.youtube.com/watch?v=xND0t1pr3KY}
			\item Learn-C Tutorial (interactive): \href{https://www.learn-c.org/}{https://www.learn-c.org/}.
			\item Amazing C guide for anything in this course --- and much more: \href{https://beej.us/guide/bgc/}{https://beej.us/guide/bgc/}.
			\begin{itemize}
				\item You might also fancy the corresponding networks programming using C, by Beej: \href{https://beej.us/guide/bgnet/}{https://beej.us/guide/bgnet/}.
			\end{itemize}
		\end{itemize}
	\end{frame}
	
	\section{Fun Time!}\label{sec:fun-time}
	
	\sectionframe
	
	\begin{frame}{In--class Exercise \#001}
		Write a C++ program that:
		\begin{itemize}
			\item asks the user for a positive integer number, $n$;
			\item checks if the number is even or odd, and;
			\item prints on screen \texttt{even} if the number is even, \texttt{odd} otherwise.
		\end{itemize}
	\end{frame}

	\begin{frame}{In--class Exercise \#002}
		A PC manufacturer has the following retail pricing catalogue:
		\begin{itemize}
			\item For orders with at most 100 PCs, unit cost is 560\$ each.
			\item For orders with at most 250 PCs, the first 100 are priced as above and the rest at 480\$ each.
			\item For orders with at most 400 PCs, the first 250 are priced as above and the rest at 400\$ each.
			\item For orders above 400 PCs, the first 400 are priced as above and the rest at 320\$ each.
		\end{itemize}
		Write a C++ program that accepts the number of PCs ordered by some customer and computes and prints on screen to total cost of that order.
	\end{frame}

	\begin{frame}{In--class Exercise \#003}
		Write a C++ program that:
		\begin{itemize}
			\item asks the user for a positive integer number, $n$;
			\item checks if the number is prime or not, and;
			\item prints on screen \texttt{prime} if the number is prime, \texttt{composite} otherwise.
		\end{itemize}
		As a reminder, a positive integer, $n$, is said to be prime if the following conditions hold (both of them):
		\begin{itemize}
			\item $n>1$.
			\item The only divisors of $n$ are $1$ and $n$.
		\end{itemize}
		So, 2, 7, 13 and 19 are some primes while 4, 15 and 21 are not.
	\end{frame}

	\begin{frame}{In--class Exercise \#004}%
		\begin{minipage}[t]{0.40\textwidth}%
			\vspace{0pt}\raggedright%
			Consider the C code shown right.
			\begin{itemize}
				\item Without compiling and executing it, what do you expect it to do?
				\item Compile and run that program. What did it print?
				\item Can you explain it?
			\end{itemize}
		\end{minipage}\hfill
		\begin{minipage}[t]{0.57\textwidth}%
			\vspace{0pt}\raggedleft%
			\lstinputlisting[language=C]{../source/exercise004.c}
		\end{minipage}
	\end{frame}

	\begin{frame}{In--class Exercise \#005}%
		\begin{minipage}[t]{0.40\textwidth}%
			\vspace{0pt}\raggedright%
			Consider the C code shown right.
			\begin{itemize}
				\item Without compiling and executing it, what do you expect it to do?
				\item Compile and run that program. What did it print?
				\item Can you explain it?
			\end{itemize}
		\end{minipage}\hfill
		\begin{minipage}[t]{0.57\textwidth}%
			\vspace{0pt}\raggedleft%
			\lstinputlisting[language=C]{../source/exercise005.c}
		\end{minipage}
	\end{frame}
	
	\begin{frame}{Homework}%
		Complete all exercises and problems in MIT's C++ course first assignment, found here:
		\begin{center}%
			\vspace{-1.7\topsep}%
			\small%
			\href{https://ocw.mit.edu/courses/6-096-introduction-to-c-january-iap-2011/resources/mit6_096iap11_assn01/}{\texttt{https://ocw.mit.edu/courses/6-096-introduction-to-c-january-iap-2011/\\resources/mit6\_096iap11\_assn01/}}%
		\end{center}
		For your convenience, you can also find the assignment file in this lecture's materials, at: \ohref{../homework/MIT6-096IAP11-assn01.pdf}.
		Submit all your work in the online form below as a single \texttt{.zip} file:
		\begin{center}
			\ohref{https://forms.gle/rSq3VSpcouRAVjqMA}
		\end{center}
		or via email at: \texttt{v.markos@mc-class.gr}.
	\end{frame}

	\begin{frame}{Any Questions?}
		\begin{minipage}{0.35\textwidth}
			\raggedright
			Do not forget to fill in the questionnaire shown right!
		\end{minipage}\hfill
		\begin{minipage}{0.58\textwidth}
			\vspace{0pt}
			\raggedleft
			\includegraphics[scale=0.4]{./assets/post_lesson_assessment.png}
			\centering
			\ohref{https://forms.gle/dKSrmE1VRVWqxBGZA}
		\end{minipage}
	\end{frame}
	
\end{document}