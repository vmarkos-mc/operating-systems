% !TeX spellcheck = en_GB
% !TeX TS-program = xelatex
\documentclass[aspectratio=169, 12pt]{beamer}
\usefonttheme{professionalfonts}
\usefonttheme{serif}
\usepackage[T1]{fontenc}
\usepackage{fontspec-xetex}

\usepackage{booktabs}
\usepackage{listings}
\usepackage{subcaption}
\setmainfont{Lato}

% Local configuration
\renewcommand{\figurename}{}
\DeclareCaptionFormat{custom}
{%
	\tiny #3
}
\captionsetup{format=custom}

% Title stuff
\title{Systems Programming}
\subtitle{A (Soft) Introduction to C, Part II}
\date{Week 06}
\author{Vassilis Markos, Mediterranean College}

\usetheme{streamline}

% Local Commands
\newcommand{\ohref}[1]{\href{#1}{\texttt{#1}}}

% Code listings

\definecolor{codegreen}{rgb}{0,0.6,0}
\definecolor{codegray}{rgb}{0.5,0.5,0.5}
\definecolor{codepurple}{rgb}{0.58,0,0.82}
\definecolor{backcolour}{rgb}{0.95,0.95,0.92}

\lstdefinestyle{mystyle}{
	backgroundcolor=\color{backcolour},   
	commentstyle=\color{codegreen},
	keywordstyle=\color{magenta},
	numberstyle=\tiny\color{codegray},
	stringstyle=\color{codepurple},
	basicstyle=\ttfamily\footnotesize,
	breakatwhitespace=false,         
	breaklines=true,                 
	captionpos=b,                    
	keepspaces=true,                 
	numbers=left,                    
	numbersep=5pt,                  
	showspaces=false,                
	showstringspaces=false,
	showtabs=false,                  
	tabsize=2
}

\lstset{style=mystyle}

% Exercise number

\makeatletter
\newcommand{\arabicthree}[1]{\expandafter\@arabicthree\csname c@#1\endcsname}
\newcommand{\@arabicthree}[1]{\ifnum #1<100 0\fi\ifnum #1<10 0\fi\number#1}
\makeatother

\newcounter{exno}
\setcounter{exno}{0}

\newcommand{\exno}{\stepcounter{exno}In--class Exercise \#\arabicthree{exno}}

\begin{document}

	\begin{frame}
		\titlepage
	\end{frame}

	\begin{frame}{Contents}
		\tableofcontents
	\end{frame}

	\section{C Functions}
	
	\sectionframe

	\begin{frame}{Compiling Stuff\ldots}
		\begin{figure}[!htb]
			\centering
			\includegraphics[height=0.5\textheight]{./assets/compiling.png}
			\caption{Breaking news from Developers' Land. Source: \ohref{https://xkcd.com/303/}.}
		\end{figure}
	\end{frame}
	
	\begin{frame}{Writing Some Code}
		What does this C++ code compute?
		\scalebox{0.7}{%
		\lstinputlisting[language=C]{../source/someCode.c}}
	\end{frame}
	
	\begin{frame}{How Can We Repeat This Twice?}
		\scalebox{0.59}{%
			\lstinputlisting[language=C]{../source/someNaiveCode.c}}
	\end{frame}
	
	\begin{frame}{What About Loops?}
		\begin{center}
			\scalebox{0.7}{%
				\lstinputlisting[language=C]{../source/whatAboutLoops.c}}
		\end{center}
	\end{frame}
	
	\begin{frame}{Abstraction Through Functions}
		\begin{center}
			\scalebox{0.59}{%
			\lstinputlisting[language=C]{../source/functionAbstraction.c}}
		\end{center}
	\end{frame}
	
	\begin{frame}{Why Use Functions?}
		\begin{itemize}
			\item \textbf{Code Maintenance:} It is way easier to maintain your code if all functionality that is intended to be reused is defined and kept at one place.
			\item \textbf{Code Distribution:} Imagine reading a cryptic project where everything is defined at difficult to spot places. Avoid this for your projects as much as possible.
			\item \textbf{Debugging:} The more concise your code the easiest to understand and, consequently, the easiest to debug.
		\end{itemize}
	\end{frame}
	
	\begin{frame}{A Brief Reality Check}
		\begin{figure}
			\includegraphics[scale=0.55]{./assets/code_lifespan.png}
			\caption{Another day in office. Source: \ohref{https://xkcd.com/2730/}}
		\end{figure}
	\end{frame}
	
	\begin{frame}{Function Terms}
		\begin{itemize}
			\item \textbf{Function signature:} The part of the function definition that includes:
			\begin{itemize}
				\item Its name, which may be any alhpanumeric string starting with a letter.
				\item Its return type, which is any C valid type.
				\item Its arguments, which is a list of typed variable names.
			\end{itemize}
			\item \textbf{Return value:} Every function can return up to one variable, whose type should match the function's return type.
		\end{itemize}
	\end{frame}
	
	\begin{frame}{What's Wrong Here?}
		Execute the following program. What goes wrong?
		
		%		\begin{center}
		\scalebox{0.85}{%
			\lstinputlisting[language=C]{../source/problem_001.c}}
		%		\end{center}
	\end{frame}
	
	\begin{frame}{What's Wrong Here?}
		Execute the following program. What goes wrong?
		\scalebox{0.85}{%
			\lstinputlisting[language=C]{../source/problem_002.c}}
	\end{frame}
	
	\begin{frame}{What's Not Wrong Here?}
		Execute the following program. Why does it work?
		\scalebox{0.85}{%
			\lstinputlisting[language=C]{../source/problem_003.c}}
	\end{frame}
	
	\begin{frame}{What's Wrong Here?}
		Execute the following program. What goes wrong?
		\scalebox{0.85}{%
			\lstinputlisting[language=C]{../source/problem_004.c}}
	\end{frame}
	
	\begin{frame}{What's Not Wrong Here?}
		Execute the following program. Why does it work?
		\scalebox{0.85}{%
			\lstinputlisting[language=C]{../source/problem_005.c}}
	\end{frame}
	
	\begin{frame}{Important Notes On Functions}
		\begin{itemize}
			\item Functions that do not return a value should be declared as \texttt{void}.
			\item C will try to cast types whenever it can. For instance, C can cast doubles to integers, so it will, if prompted to.
			\begin{itemize}
				\item We will discuss about all of that stuff in more detail throughout this course.
				\item If you feel an urge to learn more, you can look up \texttt{rvalue} and \texttt{lvalue}.
			\end{itemize}
			\item Always remember that characters in C can also be treated as integers, with the corresponding pros and cons.
		\end{itemize}
	\end{frame}
	
	\begin{frame}{Will This Work?}
		Will this work? Answer before executing the program first!
		
		\scalebox{0.60}{%
			\lstinputlisting[language=C]{../source/foo.c}}
	\end{frame}
	
	\begin{frame}{Function Overloading}
		\begin{itemize}
			\item Many languages allow us to define functions so long as they have a different signature.
			\item This means that two functions sharing the same name should either accept different types / number of arguments and / or have a different return type.
			\item \textbf{C does not natively support function overloading!}
			\item There are three ways to bypass this:
			\begin{itemize}
				\item Use different names for each overloaded version (\ldots).
				\item Use the C11 (or newer) \texttt{\_Generic} compile--time operator (\href{https://port70.net/~nsz/c/c11/n1570.html\#6.5.1}{https://port70.net/~nsz/c/c11/n1570.html\#6.5.1}).
				\item Use \texttt{void} pointers and run all sort of risks (what we commonly do).
			\end{itemize}
		\end{itemize}
	\end{frame}
	
	\section{Fun Time!}
	
	\sectionframe
	
	\begin{frame}{\exno}
		Write a C function that:
		\begin{itemize}
			\item takes a single integer as an argument, and;
			\item returns \texttt{1} or \texttt{0} depending on whether this number is even or odd.
		\end{itemize}
		Demonstrate the functionality of your function by properly using it in a simple C script.
	\end{frame}
	
	\begin{frame}{\exno}
		The Fibonacci numbers, $f_n$, are a sequence of integer numbers given by the following relation:
		\[f_n=f_{n-1}+f_{n-2},\quad f_0=0,\ f_1=1.\]
		That is, each term is the sum of its previous two. For instance, the first 10 Fibonacci numbers are:
		\[0,1,1,2,3,5,8,13,21,34.\]
		Write a C function that takes $n$ as input and prints the $n$--th Fioinacci number, $f_n$.
	\end{frame}
	
	\begin{frame}{\exno}
		Write a C program that:
		\begin{itemize}
			\item asks the user for consecutive positive integers (non--positive input terminates number insertion), and;
			\item computes and prints their sum and average.
		\end{itemize}
		You are required to use \textbf{at least three different functions} for your solution and explain your rationale!
	\end{frame}
	
	\begin{frame}{\exno}
		The standard deviation, $s$, of a set of $n$ numbers $x_1,x_2,\ldots,x_n$ is computed by the following formula:
		\[s=\sqrt{\frac{(x_1-\mu)^2+(x_2-\mu)^2+\cdots+(x_n-\mu)^2}{n}},\]
		where $\mu$ is the mean value of those numbers.
		
		Write a C program that asks the user for some non--zero numbers (insertion terminated by inserting 0) and computes their standard deviation. Make sure your program uses \textbf{at least two functions!}
	\end{frame}
	
	\begin{frame}{\exno}
		A string is said to be a \textbf{palindrome} if it reads the same left--to--right and right--to--left. Write a C function that:
		\begin{itemize}
			\item takes a single string as an argument, and;
			\item returns \texttt{True} or \texttt{False} depending on whether this string is a palindrome.
		\end{itemize}
		Demonstrate the functionality of your function by properly using it in a simple C script.
	\end{frame}
	
	\begin{frame}{\exno}
		One way to estimate the square root of a positive float, $a$, is to use the following method:
		\[x_n=\frac{1}{2}\left(x_{n-1}+\frac{a}{x_{n-1}}\right),\]
		where the first estimate, $x_0$, is an arbitrary positive float. We say that $x_n$ is an estimation of $\sqrt{a}$ of accuracy $\varepsilon>0$ if $\left| x_n-x_{n-1}\right|<\varepsilon$, i.e., if the two latest estimates we have made are no further apart than $\varepsilon$.
		
		Write a C function that takes $x_0$, $a$, and $\varepsilon$ as arguments and returns the corresponding estimate, $x_n$.
	\end{frame}
	
	\begin{frame}{\exno}
		The \emph{Towers of Hanoi} is a well--known puzzle where you have to move disks of different sizes one at a time from a peg to another peg with the help of an auxiliary peg and without ever moving a larger disk on top of a smaller one. You can familiarise yourselves with the game below:
		\begin{center}
			\ohref{https://www.mathsisfun.com/games/towerofhanoi.html}
		\end{center}
		Develop a C function that accepts a positive integer $n$ corresponding to the number of disks on the first peg and prints on screen the required steps to solve the problem.
	\end{frame}
	
%	\section{Pointers}\label{sec:pointers}
%	
%	\sectionframe
%	
%	\begin{frame}{Stars And C++}
%		What will this program print? (Do not execute it!)
%		\lstinputlisting[language=C]{../source/stars_001.c}
%	\end{frame}
%
%	\begin{frame}{What Are \texttt{*} and \texttt{\&}?}
%		\begin{itemize}
%			\item What is \texttt{*y} in the above program at line 9?\pause
%			\begin{itemize}
%				\item \texttt{*y} denotes the memory location of a \texttt{double} variable, \texttt{x} in our case. Such variables are called \textbf{pointers}.\pause
%			\end{itemize}
%			\item What is \texttt{*y} in the above program, at line 10?\pause
%			\begin{itemize}
%				\item \texttt{*y} denotes the content at the memory location stored by pointer \texttt{y}. Retrieving a pointer's pointed value is often called \textbf{dereferencing}.\pause
%			\end{itemize}
%			\item What is \texttt{\&x} in the above program at line 9?\pause
%			\begin{itemize}
%				\item \texttt{\&x} indicates the memory location where the content of variable \texttt{x} is stored in the computer's memory, i.e., it is a \textbf{reference} of \texttt{x}.
%			\end{itemize}
%		\end{itemize}
%	\end{frame}
%
%	\begin{frame}{Can You Predict The Output?}
%		\lstinputlisting[language=C]{../source/stars_002.c}
%	\end{frame}
%
%	\begin{frame}{Can You Predict The Output?}
%		\lstinputlisting[language=C]{../source/stars_003.c}
%	\end{frame}
%
%	\begin{frame}{Can You Predict The Output?}
%		\lstinputlisting[language=C]{../source/stars_004.c}
%	\end{frame}
%
%	\begin{frame}{Can You Predict The Output?}
%		\lstinputlisting[language=C]{../source/stars_005.c}
%	\end{frame}
%
%	\begin{frame}{Can You Predict The Output?}
%		\scalebox{0.85}{%
%		\lstinputlisting[language=C]{../source/stars_006.c}}
%	\end{frame}
%
%	\begin{frame}{Can You Predict The Output?}
%		\lstinputlisting[language=C]{../source/stars_007.c}
%	\end{frame}
%
%	\begin{frame}{A \texttt{const} Interlude}
%		\begin{minipage}{0.4\textwidth}
%			Consider the program shown right. Which of the following inserted in line 12 will raise an error?
%			\begin{enumerate}
%				\item \texttt{(*p1)++;}
%				\item \texttt{(*p2)++;}
%				\item \texttt{p1 = \&y;}
%				\item \texttt{p2 = \&y;}
%			\end{enumerate}
%		\end{minipage}\hfill
%		\begin{minipage}{0.56\textwidth}
%			\lstinputlisting[language=C]{../source/stars_008.c}
%		\end{minipage}
%	\end{frame}
%
%	\begin{frame}{Pointers To Constant Variables}
%		\begin{minipage}{0.4\textwidth}
%			In this case, pointer \texttt{p1} is pointing to a \textbf{constant} integer variable. This means that at the memory location \texttt{p1} is pointing to we can make no modifications!
%		\end{minipage}\hfill
%		\begin{minipage}{0.56\textwidth}
%			\lstinputlisting[language=C]{../source/stars_009.c}
%		\end{minipage}
%	\end{frame}
%
%	\begin{frame}{Pointers To Constant Variables}
%		\begin{minipage}{0.4\textwidth}
%			In this case, pointer \texttt{p2} is a \textbf{constant} pointer pointing to an integer variable. This means that at the memory location \texttt{p2} is pointing to we can make any modifications we want to. What we can't change is the value of the pointer itself.
%		\end{minipage}\hfill
%		\begin{minipage}{0.56\textwidth}
%			\lstinputlisting[language=C]{../source/stars_010.c}
%		\end{minipage}
%	\end{frame}
%
%	\begin{frame}{All \texttt{const}?}
%		What about this one?
%		\begin{center}
%			\texttt{const int * const ptr;}
%		\end{center}
%		In this case:\pause
%		\begin{itemize}
%			\item We cannot change where the pointer points to (rightmost \texttt{const}).\pause
%			\item We cannot change the content of the memory location the pointer points to (leftmost \texttt{const}).
%		\end{itemize}
%	\end{frame}
%
%%	\section{Arrays \& Strings}\label{sec:arrays----strings}
%%	
%%	\sectionframe
%	
%	\section{Fun Time!}\label{sec:fun-time}
%	
%	\sectionframe
%	
%	\begin{frame}{Advanced Pointer Fun}
%		While we have said enough about pointers, we have not explored pointer--land in full. The following tutorial will help you do so:
%		\begin{center}
%			\ohref{https://learnmoderncpp.com/arrays-pointers-and-loops/}
%		\end{center}
%		Follow the tutorial step--by--step and pay attention to the ``Experiments'' it asks you to execute. Write down your observations in a document, which you will share with me at the end of the class at: \texttt{v.markos@mc-class.gr}.
%	\end{frame}
%	
%	\begin{frame}{Homework}%
%		Complete all exercises and problems in MIT's C++ course second assignment, found here:
%		\begin{center}%
%			\vspace{-1.0\topsep}%
%			\small%
%			\href{https://ocw.mit.edu/courses/6-096-introduction-to-c-january-iap-2011/797ebff419fa2cc3a10af2c5f19be961_MIT6_096IAP11_assn02.pdf}{\texttt{https://ocw.mit.edu/courses/6-096-introduction-to-c-january-iap-\\2011/797ebff419fa2cc3a10af2c5f19be961\_MIT6\_096IAP11\_assn02.pdf}}%
%		\end{center}
%		For your convenience, you can also find the assignment file in this lecture's materials, at: \ohref{../homework/MIT6-096IAP11-assn02.pdf}.
%		Submit all your work in the online form below as a single \texttt{.zip} file:
%		\begin{center}
%			\ohref{https://forms.gle/rSq3VSpcouRAVjqMA}
%		\end{center}
%		or via email at: \texttt{v.markos@mc-class.gr}.
%	\end{frame}

	\begin{frame}{Any Questions?}
		\begin{minipage}{0.35\textwidth}
			\raggedright
			Do not forget to fill in the questionnaire shown right!
		\end{minipage}\hfill
		\begin{minipage}{0.58\textwidth}
			\vspace{0pt}
			\raggedleft
			\includegraphics[scale=0.4]{./assets/post_lesson_assessment.png}
			\centering
			\ohref{https://forms.gle/dKSrmE1VRVWqxBGZA}
		\end{minipage}
	\end{frame}
	
\end{document}